\documentclass[10pt]{moderncv}
\moderncvstyle{classic}
\usepackage[utf8]{inputenc}

\firstname{Carlos}

\familyname{Duelo Serrano}

\phone{+57 312 3529307} 

\email{c@cduelo.com}

\photo[70pt]{Foto_2.png}

\def\ConTeXt{%
  C%
  \kern-.0333emo%
  \kern-.0333emn%
  \kern-.0667em\TeX% 
  \kern-.0333emt}

\definecolor{see}{rgb}{0.5,0.5,0.5}% for web links 
\newcommand{\up}[1]{\ensuremath{^\textrm{\scriptsize#1}}}% for text subscripts

\begin{document}
\maketitle
%\makequote

\section{Education}

\cventry{2004--2011}{MsC in Computer Engineering}{Facultad de Infórmatica Universidad Politécnica de Madrid}{}{}{}

\cventry{2009--2010}{Computer Science}{Queen's University Belfast}{}{}{(Erasmus Scholarship 2009/10)}

\section{Relevant Experience}

\cventry{Sep 2010\\Apr 2012}{Researcher}{DATSI}{Facultad de Informática Universidad Politécnica de Madrid}{}{Colaboración en un proyecto de investigación nacional dedicado al estudio del clima. El objetivo final era que la aplicación se ejecutara en una maquina altamente paralela, utilizando las tecnologías MPI, OpenMP y Nvidia CUDA.} 

\cventry{May 2012\\Dec 2013}{Software Developer}{CeSViMa}{Facultad de Informática Universidad Politécnica de Madrid}{}{Colaboración en un proyecto de investigación nacional dedicado al estudio del clima. El objetivo final era que la aplicación se ejecutara en una maquina altamente paralela, utilizando las tecnologías MPI, OpenMP y Nvidia CUDA.  Trabajé en el grupo de Dinámica de Fluidos de la escuela de Aeronáuticos de la UPM. Desarrollando un renderizador de isosuperficies que permitía inspeccionar campos de turbulencias. La aplicación fue desarrollada en el lenguaje C++ y con la tecnología NVIDIA CUDA.}

\cventry{Jan 2014\\Jun 2015}{PhD Student}{CeSViMa}{Facultad de Informática Universidad Politécnica de Madrid}{}{Participé como estudiante de doctorado en el proyecto Europeo Human Brain Project en el equipo de Visualización. Mi cometido fue colaborara en el desarrollo del renderizador de Neuronas. Las tecnologías utilizadas en este proyecto fueron C++, NVIDIA CUDA y OpenGL.} 

\cventry{Jun 2015\\Ago 2017}{Software Developer}{NFQRisk Solutions}{}{}{Parte del equipo de desarrollo de soluciones propias de NFQRisk participé en varios proyectos relacionados con la banca financiera. Principalmente trabajé en un proyecto de cálculo de capital económico para entidades y otro para valoración de productos financieros como préstamos o depósitos. También realicé APIs RESTFul para distintas aplicaciones web. Las tecnologías utilizados fueron Java, Python, Apache Spark, TomCat, Oracle SQL, MongoDB.} 

\cventry{Ago 2015\\Jul 2018}{DevOps}{Avature}{}{}{Formé parte de equipo de DevOps de Avature donde principalmente me dediqué al desarrollo de aplicaciones internas para realizar despliegues, herramientas internas para desarrolladores para integración continua. La mayor parte del esfuerzo la dediqué al la aplicación que mantenía toda la replicación de repositorios Git de Avature.} 

\section{Idioms}

\cvlanguage{Spanish}{Native}{}
\cvlanguage{English}{Professional working proficiency}{}

\section{Skills}

\cvcomputer{Data Bases}{MySql, MongoDB}{Programing Experience}{C, C++, Java, Python, C\# .NET, OpenGL}
\cvcomputer{Tools}{Git, CMake, LaTeX Maven}{Parallel technologies}{openMP, MPI, NVIDIA CUDA, Apache Spark}

\end{document}
